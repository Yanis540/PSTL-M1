\chapter{Etude expérimentale}
\section{Génération d'instance Valide}
La génération d'instances JSON Schema vise à créer des ensembles de données valides à partir d'un schéma JSON défini. Ce processus est crucial pour divers cas d'utilisation, tels que la création de jeux de test, le remplissage de bases de données et l'exploration de l'espace de solutions défini par le schéma \cite{GENERATION}.
\subsection{Bibliothèques existantes}
\begin{itemize}
    \item [\textbullet] \textbf{json-schema-faker (JSF):} pour une génération rapide et simple de données fictives\cite{faker}.
    \item [\textbullet] \textbf{json-everything (JE):} écrite en C\#, est extension de \textbf{System.Text.Json}, limitée en termes d'expressivité sur la partie JSON\cite{everything}.
    \item [\textbullet] \textbf{json-data-generator (JDG):} pour une prise en charge complète de JSON Schema Draft 7 et la génération de données aléatoires\cite{data_generator}.
\end{itemize}
 
\section{Approche sur les générateurs}
Notre approche consiste a faire un étude experimentale et trouver si possible, une relation entre le nombre d'erreur et la distance d'édition \textbf{TED}. La démarche est donc de prendre le \textbf{TED} de \textbf{JEDI} et de comparer la distance d'édition d'instance généré par rapport à des instances correctes déjà fournie dans un \textbf{dataset}, on calcule pour chaque générateur le nombre d'erreur de chaque instance ainsi que la taille de l'arbre, le résultat nous permettera donc de mieux comprendre comment procéder à notre réparation d'instance.
 \subsection{Résultats}

